%!TEX root = lista3.tex

{
	\DeclareDocumentCommand \ndlimitegenerico {m m m m} {
		#1_{{#2} \to {#3}} {#4}
		}
	\DeclareDocumentCommand \ndlimite {m O{\pinf} m} {
		\ndlimitegenerico{\lim}{#1}{#2}{#3}
		}

	\hfill

	\begin{enumerate}[a)]

	\item

		Dada la sucesión $(x_n)$, sea $ A = \set{a\in\R | a = \ndlimite{n_k}{x_{n_k}}} $ el conjunto de los valores de adherencia de $(x_n)$.

		Si $a\in \cierre{A}$, existe una sucesión $(a_n)$, con $a_n\in A\ \forall n\in\N$ tal que $\limite{n}{a_n} = a$. Por definición, $\forall\varepsilon>0\ \exists n_1\in\N; n>n_1\Rightarrow |a_n-a| < \frac{\varepsilon}{2}$.

		Tomando $n>n_1$, $a_n = \limite{n_k}{x_{n_k}}$ para alguna subsucesión $(x_{n_k})$ porque $a_n\in A$. Luego $\forall\varepsilon>0\ \exists n_2\in\N; n_k>n_2 \Rightarrow |x_{n_k}-a_n| < \frac{\varepsilon}{2}$.

		Para $n_0 = \max\set{n_1,n_2}$ se obtiene $\forall\varepsilon > 0\ \exists n_0\in\N; n,n_k > n_0 \Rightarrow |a_n-a| < \frac{\varepsilon}{2} \wedge |x_{n_k}-a_n| < \frac{\varepsilon}{2}$.
		Luego, $ |x_{n_k}-a| = |(x_{n_k}-a_n)+(a_n-a)| \leq |x_{n_k}-a_n| + |a_n-a| < \frac{\varepsilon}{2} + \frac{\varepsilon}{2} < \varepsilon $, lo cual concluye que $a = \limite{n_k}{x_n}$ y $a\in A$. Por tanto, $A$ es cerrado.

	\item

		Sea $X$ un conjunto numerable denso en $F$. Esto es, $X\subset F\subset\cierre{X}$. $X\subset F$ implica $\cierre{X}\subset\cierre{F} = F \Rightarrow F = \cierre{X}$.
		Siendo $X = \set{x_1,x_2,\ldots,x_n,\ldots}$, se puede construir $(y_n)$ del siguiente modo:
		$$
			y_{n(n+1)/2+r} = \left\{
			\begin{aligned}
				x_n &\text{ si } r=0 \\
				x_r &\text{ si } 0<r<n
			\end{aligned}
			\right.
		$$

		Abajo se ilustra la forma que adopta la sucesión.
		\begin{align*}
			y_1 &= x_1 \\
			y_2 &= x_1 & y_3 &= x_2 \\
			y_4 &= x_1 & y_5 &= x_2 & y_6 &= x_3 \\
			    &\vdots & &\vdots & &\vdots & &\ddots
		\end{align*}

		Dada una sucesión $(z_n)$ de elementos de $X$, es posible construir una subsucesión $(y_{n_k}) = (z_n)$.
		Si $z_m = x_r$, se toma $y_{n(n+1)/2+r}$, con $r<n$.
		Además, se tiene en cuenta que si $z_{m_1}=y_{n_1(n_1+1)/2+r_1}$ y $z_{m_2}=y_{n_2(n_2+1)/2+r_2}$, debe tomarse $n_2>n_1$.

		Luego, el conjunto de valores de adherencia de $(y_n)$ es $\cierre{X} = F$.
	
	\end{enumerate}

}