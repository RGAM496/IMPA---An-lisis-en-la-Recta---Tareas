\documentclass{article}

\usepackage[spanish]{babel}
\usepackage[utf8]{inputenc}

\usepackage{amssymb,amsmath,amsthm}
\usepackage{mathtools}
\usepackage{enumerate}
\usepackage{datetime}

%-------------------------------------------------------------------------------

\theoremstyle{definition}
\newtheorem{enunciado}{Enunciado}

\newenvironment{ejercicio}{\begin{enunciado}}{\end{enunciado}}
\newenvironment{solucion}{\begin{proof}[Solución]}{\end{proof}}

%-------------------------------------------------------------------------------

\newcommand{\N}{\mathbb{N}^*}
\newcommand{\Z}{\mathbb{Z}}
\newcommand{\Q}{\mathbb{Q}}
\newcommand{\R}{\mathbb{R}}

\newcommand{\set}[1]{\left\{#1\right\}}

\DeclareMathOperator{\card}{card}

%-------------------------------------------------------------------------------

\title{Lista 1}
%\author{Rafael Guillermo Arias Michel}
\newdate{fecha}{27}{3}{2015}
\date{\displaydate{fecha}}

%-------------------------------------------------------------------------------

\begin{document}

	\maketitle

	\begin{ejercicio}%!TEX root = lista1.tex

Probar que el conjunto de las biyecciones $f:\N\to\N$ no es numerable.\end{ejercicio}
	% \begin{solucion}%!TEX root = lista2.tex

\hfill

\begin{enumerate}[a)]
	
	\item
		Por la definición,
		$$\limite{n}{a_n}=L \implies \forall\varepsilon>0\ \exists n_0\in\N; n>n_0\Rightarrow |a_n-L|<\varepsilon .$$
		Como $\Z=\cierre{\Z}$, $L\in\Z$ y $|a_n-L|\in\Z$ por ser una resta de enteros.
		Luego, si $\varepsilon\leq 1$, $\exists n_0\in\N; n>n_0\Rightarrow |a_n-L|<1 \implies |a_n-L| = 0 \implies a_n=L\ \forall n\geq n_0+1$.
	
	\item
		Sea $A = \set{(a_n)_{n\in\N}|a_n\in\Z\ \forall n\in\N \text{ y } \limite{n}{a_n}=0}$.

		$\forall (a_n)\in A\ \exists n_a\in\N; n\geq n_a+1 \Rightarrow a_n=0$.
		Entonces, $\sum_{n=1}^{\infty}{a_n} = \sum_{n=1}^{n_a}{a_n}$ converge.
		Para cada sucesión $(a_n)$, podemos considerar sus primeros $n_a$ términos.

		Sea $p_n$ el $n$-ésimo número primo positivo. Luego, podemos definir $f:A\to\N$ y $g:\N\times\N\to\N$ del siguiente modo:
		\begin{align*}
			g(a,i) &= \left\{
				\begin{aligned}
					p_{2n-1} &\text{ si } a \geq 0 \\
					p_{2n} &\text{ si } a < 0
				\end{aligned}
				\right.
			\\
			f\big((a_n)\big) &= \prod_{i=1}^{\infty}{g(a_i,i)}
		\end{align*}

		Si $a_n = 0\ \forall n > n_a$, es evidente que $f\big((a_n)\big) = \prod_{i=1}^{n_a}{g(a_i,i)}$.
		Luego, $f\big((a_n)\big) = f\big((b_n)\big) \Leftrightarrow g(a_i,i) = g(b_i,i) \forall i\in\N \Rightarrow a_i = b_i \forall i\in\N$ por el teorema de la factorización única, lo cual concluye que $f$ es inyectiva y, por tanto, $A$ es numerable.

\end{enumerate}\end{solucion}

	\begin{ejercicio}%!TEX root = lista2.tex

Suponga que $a_n>0\ \forall n\in\N$ y $\sum_{n=1}{\infty}{a_n}$ converge. Para cada $n\in\N$, sea $r_n=\sum_{n=k}^{\infty}{a_k}$.

\begin{enumerate}
	\item Probar que $\sum_{n=1}^{\infty}{a_n/r_n}$ diverge.
	\item Probar que $\sum_{n=1}^{\infty}{a_n/\sqrt{r_n}}$ converge.
\end{enumerate}\end{ejercicio}
	% \begin{solucion}%!TEX root = lista2.tex

{
	\newcommand{\openci}[2]{\left(#1-#2,#1+#2\right)}
	Para ambas partes, tendremos en cuenta que $\exists s\in\R; s=\sum_{n=1}^{\infty}{a_n}$. Luego, $r_n = s - s_{n-1}$ y
	$$
		\limite{n}{r_n} = \limite{n}{s-s_n} = \limite{n}{s} - \limite{n}{s_n} = s-s = 0
	.$$
	Además, $r_n-r_{n+1} = a_n > 0 \Rightarrow r_n > r_{n+1}\ \forall n\in\N$. También, $r_n = a_n + a_{n+1} + \ldots > a_n$.

	\begin{enumerate}[a)]

		\item
			\begin{align*}
				\sum_{i=m}^{n}{\frac{a_i}{r_i}}
				&= \sum_{i=m}^{n}{\frac{r_i-r_{i+1}}{r_i}}
				> \sum_{i=m}^{n}{\frac{r_i-r_{i+1}}{r_m}}
				= \frac{r_m-r_{n+1}}{r_m}
				= 1 - \frac{r_{n+1}}{r_m}
				> 1 - \frac{r_n}{r_m}
			\end{align*}

			Luego:
			\begin{align*}
				\sum_{i=m}^{\infty}{\frac{a_i}{r_i}}
				&= \limite{n}{\sum_{i=m}^{n}{\frac{a_i}{r_i}}}
				= \limite{n}{\left(1-\frac{r_n}{r_m}\right)}
				= 1
			\end{align*}

			Esto significa que para cualquier $\varepsilon > 0\ \exists n_0\in\N; n>n_0 \Rightarrow \sum_{i=m_0}^{n}{\frac{a_i}{r_i}} \in\openci{1}{\varepsilon}$.
			Para todo $k\in\N$, tomamos $m_k = n_{k-1} + 1$ y hallamos $n_k\in\N$ tal que $n>n_k \Rightarrow \sum_{i=m_k}^{n}{\frac{a_i}{r_i}} \in\openci{1}{\varepsilon}$.
			Así, obtenemos una suma infinita de partes de la serie $\sum_{n=1}^{\infty}{a_n/r_n}$ (sin repetir elementos), y cada una de las partes es mayor a $1-\varepsilon$. Si hacemos $\varepsilon<\frac{1}{2}$ tenemos una suma infinita de expresiones mayores a $\frac{1}{2}$, la cual diverge.
			Con esto se concluye que $\sum_{n=1}^{\infty}{a_n/r_n}$ diverge.

		\item
			\begin{align*}
				\frac{a_n}{\sqrt{r_n}}
				&= \frac{a_n(\sqrt{r_n}+\sqrt{r_{n+1}})}{\sqrt{r_n}(\sqrt{r_n}+\sqrt{r_{n+1}})}
				= \frac{1}{\sqrt{r_n}+\sqrt{r_{n+1}}}\left( a_n + \frac{a_n\sqrt{r_{n+1}}}{\sqrt{r_n}} \right)
				\\
				&< \frac{2a_n}{\sqrt{r_n}+\sqrt{r_{n+1}}}
				= \frac{2(r_n-r_{n+1})}{\sqrt{r_n}+\sqrt{r_{n+1}}}
				= 2(\sqrt{r_n}-\sqrt{r_{n+1}})
			\end{align*}

			Si $m<n$, entonces $\sum_{i=m}^{n}\frac{a_i}{\sqrt{r_i}} < \sum_{i=m}^{n} 2(\sqrt{r_i}-\sqrt{r_{i+1}}) = 2(\sqrt{r_m}-\sqrt{r_n})$.
			Luego:
			\begin{align*}
				\sum_{i=m}^{\infty}\frac{a_i}{\sqrt{r_i}}
					&= \limite{n}{\sum_{i=m}^{n}\frac{a_i}{\sqrt{r_i}}}
					\leq \limite{n}{2(\sqrt{r_m}-\sqrt{r_n})}
				\\
				\sum_{i=m}^{\infty}\frac{a_i}{\sqrt{r_i}}
					&\leq 2\limite{n}{\sqrt{r_m}} - 2\limite{n}{\sqrt{r_n}}
					= 2\sqrt{r_m}
			\end{align*}

			En particular, para $m=1$, $\sum_{i=1}^{\infty}\frac{a_i}{\sqrt{r_i}} \leq 2\sqrt{r_1}$ converge.

	\end{enumerate}
}\end{solucion}

	\begin{ejercicio}%!TEX root = lista1.tex

La desigualdad entre la media aritmética y la geométrica vale para $n$ números reales positivos $x_1,\ldots,x_n$.
Sean $G = \sqrt[n]{x_1x_2\ldots x_n}$ y $A = \frac{x_1+x_2+\ldots+x_n}{n}$. Se tiene $G\leq A$. Esto es evidente cuando $x_1 = x_2 = \ldots x_n$.
Para probar el caso general, considere la operación que consiste en sustituir el menor de los números dados, digamos $x_i$, y el mayor de ellos, digamos $x_j$, respectivamente por $x'_i = \frac{x_i\cdot x_j}{G}$ y $x'_j = G$.
Esto no altera la media geométrica y no aumenta la media aritmética, pues, como fácilmente se ve, $x'_i+x'_j \leq x_i+x_j$.
Pruebe que, repetida esta operación un máximo de $n$ veces, obtenemos $n$ números todos iguales a $G$ y, por tanto, su media aritmética es $G$.
Como en cada operación no aumentó la media aritmética, concluya que $G\leq A$, o sea $\sqrt[n]{x_1x_2\ldots x_n} \leq \frac{x_1+x_2+\ldots+x_n}{n}$.\end{ejercicio}
	% \begin{solucion}%!TEX root = lista3.tex

{
	\newcommand{\ba}{\frac{f(b)-f(a)}{g(b)-g(a)}}
	\newcommand{\ha}{\frac{f(a+h)-f(a)}{g(a+h)-g(a)}}

	\hfill

	\begin{enumerate}[a)]
		
		\item 

			Dados $a,b\in\R$ tales que $a<b$, defínase $l(x) = f(x) + d g(x)$ de tal modo que $l(a)=l(b)$.
			De ser así, $f(a) + d g(a) = f(b) + d g(b) \Rightarrow d = \frac{f(a)-f(b)}{g(b)-g(a)}$.

			Por el teorema de Rolle, existe $c\in(a,b)$ tal que $l'(c) = 0$.
			Esto es $f'(c) + d g'(c) = 0 \Rightarrow \frac{f'(c)}{g'(c)} = -d = \ba$.

			Esto es equivalente a decir, dados $a\in\R$ y $h>0$, existe $\theta\in(0;1)$ tal que 
			\begin{align} \label{eq:valormediogeneral}
				\ha = \frac{f'(a+\theta h)}{g'(a+\theta h)} .
			\end{align}

			Sabiendo $\limite{x}{f(x)} = \limite{x}{g(x)} = 0$, obtenemos que 
			\begin{align} \label{eq:limha}
				\limite{h}{\ha} = \limite{h}{\frac{0-f(a)}{0-f(b)}} = \frac{f(a)}{f(b)} .
			\end{align}

			Por hipótesis, $\forall\varepsilon>0\ \exists A>0; x>A\Rightarrow f'(x)/g'(x) \in \openci{L}{\frac{\varepsilon}{2}}$.
			Si $a>A$, de \eqref{eq:valormediogeneral} obtenemos 
			$$ \left| \ha - L \right| < \frac{\varepsilon}{2}.$$

			De \eqref{eq:limha} deducimos
			$$ \forall\varepsilon>0\ \exists H>0; h>H \rightarrow \left| \ha - \frac{f(a)}{g(a)} \right| < \frac{\varepsilon}{2} .$$

			Luego, con $a>A$ y $h>H$,
			\begin{align*}
				\left| \frac{f(a)}{g(a)} - L \right|
					&= \left| \left(\frac{f(a)}{g(a)} - \ha \right) + \left(\ha - L \right) \right| \\
					&\leq \left| \frac{f(a)}{g(a)} - \ha \right| + \left|\ha - L \right| \\
					& < \frac{\varepsilon}{2} + \frac{\varepsilon}{2} = \varepsilon .
			\end{align*}

			Por tanto, $\limite{x}{\frac{f(x)}{g(x)} = L}$.

		\item

			Demostremos que para cualquier $\varepsilon>0$ existen $\varepsilon_1,\varepsilon_2>0$, con $\varepsilon_2<1$ tales que
			\begin{align} \label{rel:chanta}
				L-\varepsilon < \frac{L-\varepsilon_1}{1+\varepsilon_2} , && L+\varepsilon > \frac{L+\varepsilon_1}{1-\varepsilon_2} .
			\end{align}
			Resolviendo las desigualdades, obtenemos $\varepsilon_1+(L-\varepsilon)\varepsilon_2 < \varepsilon$ y $\varepsilon_1+(L+\varepsilon)\varepsilon_2 < \varepsilon$; y adicionándolas deducimos $\varepsilon_1+L\varepsilon_2 < \varepsilon$.
			Si $\varepsilon_1=\varepsilon_2$ tenemos $(L+1)\varepsilon_1<\varepsilon$.
			En ese caso, si $L+1\leq 0$, cualquier $\varepsilon_1<1$ es suficiente. En cambio, si $L+1>0$, necesitamos $\varepsilon_1<\min\set{1,\frac{\varepsilon}{L+1}}$.

			Sean $x,h\in\R^+$. Luego, por \eqref{eq:valormediogeneral}, existe $\theta\in(0;1)$ tal que 
			\begin{align*}
				\frac{f(x+h)-f(x)}{g(x+h)-g(x)} = \frac{f'(x+\theta h)}{g'(x+\theta h)} .
			\end{align*}
			Obsérvese además que 
			\begin{align*}
				\frac{f(x+h)-f(x)}{g(x+h)-g(x)} = \frac{f(x)\left[\frac{f(x+h)}{f(x)}-1\right]}{g(x)\left[\frac{g(x+h)}{g(x)}-1\right]} .
			\end{align*}

			Para cualquier $\varepsilon>0$, existe $\varepsilon_1$ tal que se cumplan las dos desigualdades mencionadas en \eqref{rel:chanta}.
			Además, existe $\delta_1$ tal que $0<x<\delta_1\Rightarrow f'(x)/g'(x)\in\openci{L}{\varepsilon_1}$.
			Entonces, si tomamos $x,h\in\R+$ tales que $x+h<\delta_1$ deducimos que
			\begin{align} \label{rel:cotaL}
				L - \varepsilon_1 \leq
				\frac{f(x+h)-f(x)}{g(x+h)-g(x)} = \frac{f'(x+\theta h)}{g'(x+\theta h)} = \frac{f(x)\left[\frac{f(x+h)}{f(x)}-1\right]}{g(x)\left[\frac{g(x+h)}{g(x)}-1\right]}
				\leq L + \varepsilon_1 .
			\end{align}

			Fijando la $h$, además, podemos observar que
			\begin{align*}
				\limite{x}[0]{
					\frac
						{\frac{f(x+h)}{f(x)}-1}
						{\frac{g(x+h)}{g(x)}-1}
					}
				= \frac{0-1}{0-1} = 1 .
			\end{align*}
			Esto es lo mismo que decir que, para $x$ suficientemente pequeña (digamos, $0<x<\delta_2$), 
			\begin{align} \label{rel:cota1}
				\frac
					{\frac{f(x+h)}{f(x)}-1}
					{\frac{g(x+h)}{g(x)}-1}
				\in\openci{1}{\varepsilon_1} .
			\end{align}

			Combinando \eqref{rel:cotaL} y \eqref{rel:cota1} tenemos
			\begin{align*}
				L-\varepsilon_1 < \frac{f(x)}{g(x)} (1+\varepsilon_1) 
				, &&
				\frac{f(x)}{g(x)} (1-\varepsilon_2) < L+\varepsilon_1
				.
			\end{align*}

			Inmediatamente deducimos que
			\begin{align*}
				L-\varepsilon < \frac{L-\varepsilon_1}{1+\varepsilon_1}
				< \frac{f(x)}{g(x)} <
				\frac{L+\varepsilon_1}{1-\varepsilon_1} < L+\varepsilon
			\end{align*}
			y concluimos que $\limite{x}[0]{\frac{f(x)}{g(x)}} = L$.

	\end{enumerate}
}\end{solucion}

	\begin{ejercicio}%!TEX root = lista1.tex

Un conjunto $G$ denúmeros reales es un \emph{grupo aditivo} cuando $0\in G$ y $x,y\in G \Rightarrow x-y\in G$.
Entonces, $x\in G \Rightarrow -x\in G$ y $x,y\in G \Rightarrow x+y\in G$.
Sea entonces $G \subset \R$ un grupo aditivo de números reales. Indiquemos con $G^+$ al conjunto de números reales positivos pertenecientes a $G$.
Exceptuando el caso trivial $G=\set{0}$, $G^+$ es no vacío. Supongamos entonces $G\neq \set{0}$. Pruebe que:

\begin{enumerate}[i)]
	\item Si $\inf G^+ = 0$, entonces $G$ es denso en $\R$.
	\item Si $\inf G^+ = a > 0$, entonces $a\in G^+$ y $G = \set{\pm a, \pm 2a, \ldots}$.
	\item Concluya que, si $\alpha\in\R$ es irracional, los números reales de la forma $m+n\alpha$, con $m,n\in\Z$, constituyen un subconjunto denso en $\R$.
\end{enumerate}\end{ejercicio}
	% \begin{solucion}%!TEX root = lista2.tex

{
	\newcommand{\F}{\bigcup_{n\in\N}{F_n}}
	\newcommand{\I}{\bigcap_{n\in\N}{I_n}}
	\newcommand{\openci}[2]{\left(#1-#2,#1+#2\right)}

	Primero, demostraremos que, si $C$ es un conjunto cerrado con interior vacío y $A$ cualquier conjunto abierto, entonces existe un intervalo abierto $I\in A$ tal que $I\cap C = \emptyset$.

	Supongamos, por absurdo, que todo intervalo abierto $I\in A$ tiene al menos un punto en $C$. Como $A$ es abierto, $c\in A \Rightarrow \exists \varepsilon_c > 0; \openci{c}{\varepsilon_c} \subset A$.
	Como $\interior(C) = \emptyset$, no existe un intervalo abierto completamente contenido en $C$ (sino, todo punto en el interior del invervalo sería un punto en el interior de $C$).
	Entonces, $\exists a\in \openci{c}{\varepsilon_c}-C$.
	Sea $\varepsilon$ tal que $\openci{a}{\varepsilon}\subset\openci{c}{\varepsilon_c}$.
	Ahora, construimos la sucesión $(a_n)$ tal que $a_n\in \openci{a}{\varepsilon/n}\cap C$.
	Inmediatamente, $\limite{n}{a_n} = a$, pero como $C$ es cerrado y $a_n\in C\ \forall n\in\N$, se llega a la contradicción de que $a\in C$.
	Así concluimos que $A$ tiene al menos un intervalo abierto sin puntos en común con $C$.

	Sabemos ahora que $A$ contiene un intervalo abierto $I_1$ tal que $I_1\cap F_1 = \emptyset$.
	Inductivamente, $I_n$ contiene un intervalo abierto $I_{n+1}$ tal que $I_{n+1}\cap F_{n+1} = \emptyset$ para toda $n\in\N$.
	Luego, $\I\cap\F=\emptyset$. Como $\I\neq\emptyset$, $\exists c\in\I$ y $c\notin\F$.

	Obtenemos así que cualquier conjunto abierto $A$ tiene al menos un punto no perteneciente a $\F$, por tanto, para $x\in\F$, $\nexists\varepsilon; \openci{x}{\varepsilon}\subset\F$, lo cual concluye que $\interior\left(\F\right) = \emptyset$.
}\end{solucion}

	\begin{ejercicio}%!TEX root = lista3.tex

Decimos que $f:X\to\R$ es \emph{semicontinua superiormente} si
$\forall a\in X, \forall \varepsilon>0\ \exists \delta>0; x\in\openci{a}{\delta} \cap X \Rightarrow f(x) < f(a) + \varepsilon$.
Pruebe que si $K\subset\R$ es compacto con $K\neq\emptyset$ y $f:K\to\R$ es semicontinua superiormente entonces existe $b\in K$ tal que $f(x)\leq f(b)\ \forall x\in K$.\end{ejercicio}
	% \begin{solucion}%!TEX root = lista2.tex

{
	\newcommand{\openci}[2]{\left(#1-#2,#1+#2\right)}
	\newcommand{\li}[1]{\mathcal{I}_{#1}}
	\newcommand{\ls}[1]{\mathcal{S}_{#1}}

	\newcommand\numberthis{\addtocounter{equation}{1}\tag{\theequation}}


	Sea $g:[0;\pinf)\to\R$ definida como $g(x)=f(x+1)-f(x)$.
	Luego, $\limite{x}{g(x)} = \limite{x}{\left[f(x+1)-f(x)\right]} = L$.
	Por definición, $\forall\varepsilon>0\ \exists A>0; x>A\Rightarrow g(x)\in\openci{L}{\varepsilon}$.

	Sea $x>A$. Existe $r\in(0;1]$ tal que $x-A-r\in\Z$. Luego,
	\begin{align*}
		\frac{f(x)}{x}
		&= \frac{\sum_{y=A+r}^{x-1}\left[f(y+1)-f(y)\right]+f(A+r)}{x}
		\\
		&= \frac{\sum_{y=A+r}^{x-1}g(y)+f(A+r)}{x}
	\end{align*}

	De $L-\varepsilon<g(y)<L+\varepsilon\ \forall y>A$ se deduce
	$$ (x-A-r)(L-\varepsilon) < \sum_{y=A+r}^{x-1}g(y) < (x-A-r)(L+\varepsilon) .$$

	Luego,
	\begin{align} \label{ineq:entorno}
		\frac{(x-A-r)(L-\varepsilon)-f(A+r)}{x} < &\frac{f(x)}{x} < \frac{(x-A-r)(L+\varepsilon)-f(A+r)}{x} \notag
		\\
		\frac{(x-A-r)}{x}\cdot(L-\varepsilon)-\frac{f(A+r)}{x} < &\frac{f(x)}{x} < \frac{(x-A-r)}{x}\cdot(L+\varepsilon)-\frac{f(A+r)}{x}
	\end{align}
	para cualquier $x>A$.

	$f$ es una función acotada en cada invervalo acotado.
	Entonces, también lo es $f(x)/x$.
	Por tanto, para todo $a\in\R$, existen $\li{a}:=\limiteinferior{x}[a]{\frac{f(x)}{x}}$ y $\ls{a}:=\limitesuperior{x}[a]{\frac{f(x)}{x}}$

	Volviendo a la desigualdad \eqref{ineq:entorno}, sea $(x_n)$ una secuencia de números en $(A,\pinf)$ tal que $\limite{n}{x_n}=\pinf$ $\limite{n}{\frac{f(x_n)}{x_n}}=\li{\pinf}$.
	Luego,
	\begin{align*} \label{ineq:li}
		\frac{f(x_n)}{x_n}
			&> \frac{(x_n-A-r)}{x_n}\cdot(L-\varepsilon)-\frac{f(A+r)}{x_n}
			\ \forall n\in\N
		\\
		\limite{n}{\frac{f(x_n)}{x_n}}
			&\geq \limite{n}{\left[\frac{(x_n-A-r)}{x_n}\cdot(L-\varepsilon)-\frac{f(A+r)}{x_n}\right]}
		\\
		\li{\pinf}
			&\geq \limite{n}{\frac{(x_n-A-r)}{x_n}}\cdot\limite{n}{(L-\varepsilon)}-\limite{n}{\frac{f(A+r)}{x_n}}
		\\
		\li{\pinf}
			&\geq L-\varepsilon .\numberthis
	\end{align*}

	Análogamente, tomando otra secuencia $(x_n)$ de números en $(A,\pinf)$ tal que $\limite{n}{x_n}=\pinf$ $\limite{n}{\frac{f(x_n)}{x_n}}=\ls{\pinf}$ obtenemos
	\begin{align*} \label{ineq:ls}
		\ls{\pinf}
			&\leq L+\varepsilon .\numberthis
	\end{align*}

	De \eqref{ineq:li} y \eqref{ineq:ls} deducimos
	$ L-\varepsilon \leq \li{\pinf} \leq \ls{\pinf} \leq L+\varepsilon \ \forall\varepsilon>0 $.
	Luego, $$ L \leq \li{\pinf} \leq \ls{\pinf} \leq L \Rightarrow \li{\pinf} = \ls{\pinf} = L \Rightarrow \limite{x}{\frac{f(x)}{x}} = L .$$
}\end{solucion}

	\begin{ejercicio}%!TEX root = lista1.tex

Sea $(a_n)$ una secuencia decreciente (no estrictamente) con $\lim a_n = 0$.
Demostrar que la serie $\sum a_n$ converge si y solamente si $\sum 2^n a_{2^n}$ converge.\end{ejercicio}
	% \begin{solucion}%!TEX root = lista1.tex

Para demostrar la bicondicional, se considerarán los dos casos presentados a continuación.

~\subparagraph{Caso 1:} $\sum a_n = s$ converge.

	Como $a_n$ es decreciente, puede observarse que $2^{n-1} a_{2^n} \leq a_{2^{n-1}+1} + \ldots + a_{2^n} := b_n$. Se construye así la sucesión $(b_n)$. Es evidente que $\sum b_n = \sum a_n = s$. Como $2^na_{2^n} \leq 2b_n \forall n\in\N$, por el criterio de comparación, $\sum 2^na_{2^n}$ converge.

~\subparagraph{Caso 2:} $\sum 2^na_{2^n} = t$ converge.

	Obsérvese que $2^na_{2^n} \geq a_{2^n} + \ldots + a_{2^{n+1}-1} := c_n$.
	Al construir la sucesión $(c_n)$, se observa que $\sum c_n = \sum a_n - a_1 \Rightarrow \sum a_n = \sum c_n + a_1$.
	Como $c_n \leq 2^na_{2^n}$, por el criterio de comparación, $\sum c_n = c$ converge. Luego, $\sum a_n = c + a_1$, y también converge.\end{solucion}

\end{document}