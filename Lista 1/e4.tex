%!TEX root = lista1.tex

Un conjunto $G$ denúmeros reales es un \emph{grupo aditivo} cuando $0\in G$ y $x,y\in G \Rightarrow x-y\in G$.
Entonces, $x\in G \Rightarrow -x\in G$ y $x,y\in G \Rightarrow x+y\in G$.
Sea entonces $G \subset \R$ un grupo aditivo de números reales. Indiquemos con $G^+$ al conjunto de números reales positivos pertenecientes a $G$.
Exceptuando el caso trivial $G=\set{0}$, $G^+$ es no vacío. Supongamos entonces $G\neq \set{0}$. Pruebe que:

\begin{enumerate}[i)]
	\item Si $\inf G^+ = 0$, entonces $G$ es denso en $\R$.
	\item Si $\inf G^+ = a > 0$, entonces $a\in G^+$ y $G = \set{\pm a, \pm 2a, \ldots}$.
	\item Concluya que, si $\alpha\in\R$ es irracional, los números reales de la forma $m+n\alpha$, con $m,n\in\Z$, constituyen un subconjunto denso en $\R$.
\end{enumerate}