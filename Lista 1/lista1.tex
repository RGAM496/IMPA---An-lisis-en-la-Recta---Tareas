\documentclass{article}

\usepackage[spanish]{babel}
\usepackage[utf8]{inputenc}

\usepackage{amssymb,amsmath,amsthm}
\usepackage{mathtools}
\usepackage{enumerate}
\usepackage{datetime}

%-------------------------------------------------------------------------------

\theoremstyle{definition}
\newtheorem{enunciado}{Enunciado}

\newenvironment{ejercicio}{\begin{enunciado}}{\end{enunciado}}
\newenvironment{solucion}{\begin{proof}[Solución]}{\end{proof}}

%-------------------------------------------------------------------------------

\newcommand{\N}{\mathbb{N}^*}
\newcommand{\Z}{\mathbb{Z}}
\newcommand{\Q}{\mathbb{Q}}
\newcommand{\R}{\mathbb{R}}

\newcommand{\set}[1]{\left\{#1\right\}}

\DeclareMathOperator{\card}{card}

%-------------------------------------------------------------------------------

\title{Problemas Resueltos}
\author{Rafael Guillermo Arias Michel}
\newdate{fecha}{27}{1}{2015}
\date{\displaydate{fecha}}

%-------------------------------------------------------------------------------

\begin{document}

	\maketitle

	\begin{ejercicio}%!TEX root = lista3.tex

Decimos que $a\in\R$ es un \emph{valor de adherencia} de la secuencia $(x_n)_{n\geq 1}$ si existe alguna subsecuencia $(x_{n_k})_{k\geq 1}$ tal que $a = \limite{k}{x_{n_k}}$.

\begin{enumerate}[a)]
	
	\item Dada una secuencia $(x_n)_{n\geq 1}$, pruebe que el conjunto de susvalores de adherencia es cerrado.
	
	\item Dado un conjunto cerrado $F\subset\R$, pruebe que existe una secuencia $(x_n)_{n\geq 1}$ cuyo conjunto de valores de adherencia es $F$.

\end{enumerate}\end{ejercicio}
	\begin{solucion}%!TEX root = lista2.tex

\hfill

\begin{enumerate}[a)]
	
	\item
		Por la definición,
		$$\limite{n}{a_n}=L \implies \forall\varepsilon>0\ \exists n_0\in\N; n>n_0\Rightarrow |a_n-L|<\varepsilon .$$
		Como $\Z=\cierre{\Z}$, $L\in\Z$ y $|a_n-L|\in\Z$ por ser una resta de enteros.
		Luego, si $\varepsilon\leq 1$, $\exists n_0\in\N; n>n_0\Rightarrow |a_n-L|<1 \implies |a_n-L| = 0 \implies a_n=L\ \forall n\geq n_0+1$.
	
	\item
		Sea $A = \set{(a_n)_{n\in\N}|a_n\in\Z\ \forall n\in\N \text{ y } \limite{n}{a_n}=0}$.

		$\forall (a_n)\in A\ \exists n_a\in\N; n\geq n_a+1 \Rightarrow a_n=0$.
		Entonces, $\sum_{n=1}^{\infty}{a_n} = \sum_{n=1}^{n_a}{a_n}$ converge.
		Para cada sucesión $(a_n)$, podemos considerar sus primeros $n_a$ términos.

		Sea $p_n$ el $n$-ésimo número primo positivo. Luego, podemos definir $f:A\to\N$ y $g:\N\times\N\to\N$ del siguiente modo:
		\begin{align*}
			g(a,i) &= \left\{
				\begin{aligned}
					p_{2n-1} &\text{ si } a \geq 0 \\
					p_{2n} &\text{ si } a < 0
				\end{aligned}
				\right.
			\\
			f\big((a_n)\big) &= \prod_{i=1}^{\infty}{g(a_i,i)}
		\end{align*}

		Si $a_n = 0\ \forall n > n_a$, es evidente que $f\big((a_n)\big) = \prod_{i=1}^{n_a}{g(a_i,i)}$.
		Luego, $f\big((a_n)\big) = f\big((b_n)\big) \Leftrightarrow g(a_i,i) = g(b_i,i) \forall i\in\N \Rightarrow a_i = b_i \forall i\in\N$ por el teorema de la factorización única, lo cual concluye que $f$ es inyectiva y, por tanto, $A$ es numerable.

\end{enumerate}\end{solucion}

	\begin{ejercicio}%!TEX root = lista3.tex

Sea $(x_n)_{n\geq 1}$ una secuencia acotada.

\begin{enumerate}[a)]

	\item Pruebe que existe $\limite{n}{\left(\sup\set{x_k,k\geq n}\right)}$

	(\textbf{Observación:} definimos $\limitesuperior{n}{x_n}$ como el límite de arriba).

	\item Pruebe que $\limitesuperior{n}{x_n} = \max\set{a\in\R | a\text{ es valor de adherencia de } (x_n)}$.

\end{enumerate}\end{ejercicio}
	\begin{solucion}%!TEX root = lista1.tex

Definamos $|X| = \card(X)$.

\begin{enumerate}[a)]

	\item Demostremos por inducción.
		Supongamos inicialmente $X=\set{x_1}, Y=\set{y_1}$. Luego, $|X| = |Y| = 1 \Rightarrow |X| + |Y| = 2$.
		\begin{itemize}
			\item Si $x_1 = y_1$, $X\cup Y = \set{x_1}$ y $X\cap Y = \set{x_1}$. Luego, $|X\cup Y| + |X\cap Y| = 1 + 1 = 2$.
			\item Si $x_1 \neq y_1$, $X\cup Y = \set{x_1,y_1}$ y $X\cap Y = \emptyset$. Luego, $|X\cup Y| + |X\cap Y| = 2 + 0 = 2$.
		\end{itemize}
		Se verifica en ambos casos.

		Si $X = \set{x_1,\ldots,x_n}$ e $Y = \set{y_1,\ldots,y_m}$, $|X|=n, |Y|=m$, y luego $|X| + |Y| = n + m$. Por hipótesis de inducción, asumamos $|X\cup Y| + |X\cap Y| = n + m$.
		Sea $X' = X\cup \set{x_{n+1}},x_{n+1}\notin X$. Como $|X'| = n+1$, $|X'| + |Y| = n + m + 1$. Se dan los siguientes dos casos:
		\begin{itemize}
			\item Si $x_{n+1}\in Y$, $X'\cup Y = X \cup Y$ y $X'\cap Y = (X\cap Y) \cup \set{x_{n+1}}, x_{n+1}\notin X\cap Y$.
				Luego, $|X'\cup Y| + |X'\cap Y| = |X\cup Y| + |X\cap Y| + 1 = n + m + 1$.
			\item Si $x_{n+1}\notin Y$, $X'\cup Y = (X\cup Y) \cup \set{x_{n+1}}, x_{n+1}\notin X\cup Y$ y $X'\cap Y = X \cap Y$.
				Luego, $|X'\cup Y| + |X'\cap Y| = |X\cup Y| + 1 + |X\cap Y| = n + m + 1$.
		\end{itemize}
		Nuevamente, ambos casos verifican y queda demostrada la inducción (no se pierde generalidad al agregar un elemento a $X$ en lugar de a $Y$).

		\textbf{Observación:} Teniendo una biyección $f_n:I_n\to X$, es fácil construir una biyección $f_{n+1}:I_{n+1}\to X\cup\set{x_{n+1}}$ si $x_{n+1}\notin X$. Es por eso que añadiendo un elemento a un conjunto aumenta en 1 la cardinalidad.

	\item Utilizando lo demostrado anteriormente, deducimos:
		\begin{align}\label{eq:2_1}
			|(X\cup Y)\cup Z| + |(X\cup Y)\cap Z|
				&= |X\cup Y| + |Z| \notag \\
				&= |X| + |Y| - |X\cap Y| + |Z|
		\end{align}

		\begin{align}\label{eq:2_2}
			|(X\cup Y)\cap Z|
				&= |(X\cap Z)\cup(Y\cap Z)| \notag \\
				&= |X\cap Z| + |Y\cap Z| - |(X\cap Z)\cap(Y\cap Z)| \notag \\
				&= |X\cap Z| + |Y\cap Z| - |X\cap Y\cap Z|
		\end{align}
		
		De \eqref{eq:2_1} y \eqref{eq:2_2}:
		\begin{align*}
			|X| + |Y| + |Z|
				&= |(X\cup Y)\cup Z| + |(X\cup Y)\cap Z| + |X\cap Y| \\
				&= |X\cup Y\cup Z| + |X\cap Z| + |Y\cap Z| - |X\cap Y\cap Z| + |X\cap Y| \\
				&= |X\cup Y\cup Z| + |X\cap Z| + |X\cap Y| + |Y\cap Z| - |X\cap Y\cap Z|
		\end{align*}
		Que también puede expresarse como:
		$$|X\cup Y\cup Z| = |X| + |Y| + |Z| - |X\cap Z| - |X\cap Y| - |Y\cap Z| + |X\cap Y\cap Z|$$

	\item La forma general es:
	\begin{align*}
		|X_1\cup\ldots\cup X_n|
			&= \sum_{i=1}^{n} |X_i|
			-\ \sum_{\mathclap{1\leq i<j\leq n}}\; |X_i\cap X_j|
			+ \dots
			+ (-1)^{n-1} |X_1\cap\ldots\cap X_n|
		\\
			&= \sum_{i=1}^n \left[
				(-1)^{i-1}\ \sum_{\mathclap{1\leq n_1<\ldots<n_i\leq n}}\; |X_{n_1}\cap\ldots\cap X_{n_i}|
				\right]
	\end{align*}

	Demostremos por inducción, asumiendo que se cumple la relación para $n$ conjuntos. Entonces:
	\begin{align*}
		\left|\bigcup_{i=1}^{n+1}{X_i}\right| %|X_1\cup\ldots\cup X_{n+1}|
			&= |(X_1\cup\ldots\cup X_n)\cup X_{n+1}|
		\\
			&= |X_1\cup\ldots\cup X_n| + |X_{n+1}| - |(X_1\cup\ldots\cup X_n)\cap X_{n+1}|
		\\
			&= |X_1\cup\ldots\cup X_n| + |X_{n+1}| - |(X_1\cap X_{n+1})\cup\ldots\cup(X_n\cap X_{n+1})|
		\\	&\begin{multlined}
			= \sum_{i=1}^n \left[
				(-1)^{i-1}\ \sum_{{1\leq n_1<\ldots<n_i\leq n}}{\left|\bigcap_{j=1}^{i}{X_{n_j}}\right|}
				\right]
			+ |X_{n+1}|
		\\
			- \sum_{i=1}^n \left[
				(-1)^{i-1}\ \sum_{{1\leq n_1<\ldots<n_i\leq n}}{\left|\bigcap_{j=1}^{i}{(X_{n_j}\cap X_{n+1})}\right|}
				\right]
			\end{multlined}
		\\	&\begin{multlined}
			= \sum_{i=1}^n \left[
				(-1)^{i-1}\ \sum_{{1\leq n_1<\ldots<n_i\leq n}}{\left|\bigcap_{j=1}^{i}{X_{n_j}}\right|}
				\right]
			+ |X_{n+1}|
		\\
			- \sum_{i=1}^n \left[
				(-1)^{i-1}\ \sum_{\mathclap{1\leq n_1<\ldots<n_i\leq n}}\; |X_{n_1}\cap\ldots\cap X_{n_i}\cap X_{n+1}|
				\right]
			\end{multlined}
		\\	&\begin{multlined}
			= \sum_{i=1}^n \left[
				(-1)^{i-1}\ \sum_{{1\leq n_1<\ldots<n_i\leq n}}{\left|\bigcap_{j=1}^{i}{X_{n_j}}\right|}
				\right]
			+ |X_{n+1}|
		\\
			+ \sum_{i=2}^{n+1} \left[
				(-1)^{i-1}\ \sum_{\mathclap{1\leq n_1<\ldots<n_{i-1}\leq n}}\; |X_{n_1}\cap\ldots\cap X_{n_{i-1}}\cap X_{n+1}|
				\right]
			\end{multlined}
		\\
			&= \sum_{i=1}^{n+1} |X_i|
			+ \sum_{i=2}^n \left[
				(-1)^{i-1}\ \sum_{\mathclap{1\leq n_1<\ldots<n_i\leq n+1}}\; |X_{n_1}\cap\ldots\cap X_{n_i}|
				\right]
			+ (-1)^n \left|\bigcap_{i=1}^{n+1}{X_i}\right|
		\\
			&= \sum_{i=1}^{n+1} \left[
				(-1)^{i-1}\ \sum_{\mathclap{1\leq n_1<\ldots<n_i\leq n+1}}\; |X_{n_1}\cap\ldots\cap X_{n_i}|
				\right]
	\end{align*}

\end{enumerate}\end{solucion}

	\begin{ejercicio}%!TEX root = lista3.tex

Sean $f,g:(0;\pinf)\to\R$ funciones derivables.

\begin{enumerate}[a)]

	\item Pruebe que si existe $L = \limite{x}{\frac{f'(x)}{g'(x)}}$ y $\limite{x}{f(x)} = \limite{x}{g(x)} = 0$ entonces $\limite{x}{\frac{f(x)}{g(x)}} = L$.

	\item Pruebe que si $\limite{x}[0]{|f(x)|} = \limite{x}[0]{|g(x)|} = \pinf$ y existe $\limite{x}[0]{\frac{f'(x)}{g'(x)}} = L$ entonces $\limite{x}[0]{\frac{f(x)}{g(x)}} = L$.

\end{enumerate}\end{ejercicio}
	\begin{solucion}%!TEX root = lista3.tex

{
	\newcommand{\ba}{\frac{f(b)-f(a)}{g(b)-g(a)}}
	\newcommand{\ha}{\frac{f(a+h)-f(a)}{g(a+h)-g(a)}}

	\hfill

	\begin{enumerate}[a)]
		
		\item 

			Dados $a,b\in\R$ tales que $a<b$, defínase $l(x) = f(x) + d g(x)$ de tal modo que $l(a)=l(b)$.
			De ser así, $f(a) + d g(a) = f(b) + d g(b) \Rightarrow d = \frac{f(a)-f(b)}{g(b)-g(a)}$.

			Por el teorema de Rolle, existe $c\in(a,b)$ tal que $l'(c) = 0$.
			Esto es $f'(c) + d g'(c) = 0 \Rightarrow \frac{f'(c)}{g'(c)} = -d = \ba$.

			Esto es equivalente a decir, dados $a\in\R$ y $h>0$, existe $\theta\in(0;1)$ tal que 
			\begin{align} \label{eq:valormediogeneral}
				\ha = \frac{f'(a+\theta h)}{g'(a+\theta h)} .
			\end{align}

			Sabiendo $\limite{x}{f(x)} = \limite{x}{g(x)} = 0$, obtenemos que 
			\begin{align} \label{eq:limha}
				\limite{h}{\ha} = \limite{h}{\frac{0-f(a)}{0-f(b)}} = \frac{f(a)}{f(b)} .
			\end{align}

			Por hipótesis, $\forall\varepsilon>0\ \exists A>0; x>A\Rightarrow f'(x)/g'(x) \in \openci{L}{\frac{\varepsilon}{2}}$.
			Si $a>A$, de \eqref{eq:valormediogeneral} obtenemos 
			$$ \left| \ha - L \right| < \frac{\varepsilon}{2}.$$

			De \eqref{eq:limha} deducimos
			$$ \forall\varepsilon>0\ \exists H>0; h>H \rightarrow \left| \ha - \frac{f(a)}{g(a)} \right| < \frac{\varepsilon}{2} .$$

			Luego, con $a>A$ y $h>H$,
			\begin{align*}
				\left| \frac{f(a)}{g(a)} - L \right|
					&= \left| \left(\frac{f(a)}{g(a)} - \ha \right) + \left(\ha - L \right) \right| \\
					&\leq \left| \frac{f(a)}{g(a)} - \ha \right| + \left|\ha - L \right| \\
					& < \frac{\varepsilon}{2} + \frac{\varepsilon}{2} = \varepsilon .
			\end{align*}

			Por tanto, $\limite{x}{\frac{f(x)}{g(x)} = L}$.

		\item

			Demostremos que para cualquier $\varepsilon>0$ existen $\varepsilon_1,\varepsilon_2>0$, con $\varepsilon_2<1$ tales que
			\begin{align} \label{rel:chanta}
				L-\varepsilon < \frac{L-\varepsilon_1}{1+\varepsilon_2} , && L+\varepsilon > \frac{L+\varepsilon_1}{1-\varepsilon_2} .
			\end{align}
			Resolviendo las desigualdades, obtenemos $\varepsilon_1+(L-\varepsilon)\varepsilon_2 < \varepsilon$ y $\varepsilon_1+(L+\varepsilon)\varepsilon_2 < \varepsilon$; y adicionándolas deducimos $\varepsilon_1+L\varepsilon_2 < \varepsilon$.
			Si $\varepsilon_1=\varepsilon_2$ tenemos $(L+1)\varepsilon_1<\varepsilon$.
			En ese caso, si $L+1\leq 0$, cualquier $\varepsilon_1<1$ es suficiente. En cambio, si $L+1>0$, necesitamos $\varepsilon_1<\min\set{1,\frac{\varepsilon}{L+1}}$.

			Sean $x,h\in\R^+$. Luego, por \eqref{eq:valormediogeneral}, existe $\theta\in(0;1)$ tal que 
			\begin{align*}
				\frac{f(x+h)-f(x)}{g(x+h)-g(x)} = \frac{f'(x+\theta h)}{g'(x+\theta h)} .
			\end{align*}
			Obsérvese además que 
			\begin{align*}
				\frac{f(x+h)-f(x)}{g(x+h)-g(x)} = \frac{f(x)\left[\frac{f(x+h)}{f(x)}-1\right]}{g(x)\left[\frac{g(x+h)}{g(x)}-1\right]} .
			\end{align*}

			Para cualquier $\varepsilon>0$, existe $\varepsilon_1$ tal que se cumplan las dos desigualdades mencionadas en \eqref{rel:chanta}.
			Además, existe $\delta_1$ tal que $0<x<\delta_1\Rightarrow f'(x)/g'(x)\in\openci{L}{\varepsilon_1}$.
			Entonces, si tomamos $x,h\in\R+$ tales que $x+h<\delta_1$ deducimos que
			\begin{align} \label{rel:cotaL}
				L - \varepsilon_1 \leq
				\frac{f(x+h)-f(x)}{g(x+h)-g(x)} = \frac{f'(x+\theta h)}{g'(x+\theta h)} = \frac{f(x)\left[\frac{f(x+h)}{f(x)}-1\right]}{g(x)\left[\frac{g(x+h)}{g(x)}-1\right]}
				\leq L + \varepsilon_1 .
			\end{align}

			Fijando la $h$, además, podemos observar que
			\begin{align*}
				\limite{x}[0]{
					\frac
						{\frac{f(x+h)}{f(x)}-1}
						{\frac{g(x+h)}{g(x)}-1}
					}
				= \frac{0-1}{0-1} = 1 .
			\end{align*}
			Esto es lo mismo que decir que, para $x$ suficientemente pequeña (digamos, $0<x<\delta_2$), 
			\begin{align} \label{rel:cota1}
				\frac
					{\frac{f(x+h)}{f(x)}-1}
					{\frac{g(x+h)}{g(x)}-1}
				\in\openci{1}{\varepsilon_1} .
			\end{align}

			Combinando \eqref{rel:cotaL} y \eqref{rel:cota1} tenemos
			\begin{align*}
				L-\varepsilon_1 < \frac{f(x)}{g(x)} (1+\varepsilon_1) 
				, &&
				\frac{f(x)}{g(x)} (1-\varepsilon_2) < L+\varepsilon_1
				.
			\end{align*}

			Inmediatamente deducimos que
			\begin{align*}
				L-\varepsilon < \frac{L-\varepsilon_1}{1+\varepsilon_1}
				< \frac{f(x)}{g(x)} <
				\frac{L+\varepsilon_1}{1-\varepsilon_1} < L+\varepsilon
			\end{align*}
			y concluimos que $\limite{x}[0]{\frac{f(x)}{g(x)}} = L$.

	\end{enumerate}
}\end{solucion}

	\begin{ejercicio}%!TEX root = lista2.tex

Pruebe que si $F_n\subset\R$ es cerrado e $\interior(F_n)=\emptyset\ \forall n\in\N$ entonces $\interior\left(\bigcup_{n\in\N}{F_n}\right)=\emptyset$.\end{ejercicio}
	\begin{solucion}%!TEX root = lista1.tex

\hfill

\begin{enumerate}[i)]

	\item $\inf G^+ = 0$. Entonces, $x>0 \Rightarrow \exists y\in G^+; 0<y<x$ (sino, $0<x\leq \inf G^+$).

		Sean $r>0$ y $a_1\in G^+; a_1<r$. $b_1:=r-a_1>0$. Luego, $\exists c_1\in G^+; c<b_1/2$ y $\exists n\in\N; a_1+(n-1)c_1 < a+b_1/2 \leq a_1+nc_1 := a_2$. Sabiendo que $a_1,c_1\in G^+$, fácilmente se deduce que $a_2\in G^+$. Además, $a_2 = a_1+nc_1 = a_1+(n-1)c_1 + c_1 < a + b_1/2 + c_1 < a + b_1/2 + b_1/2 = a + b_1 = r$, y $b_2=r-a_2\leq b_1/2$.

		Iterando sucesivamente, obtenemos una sucesión $(a_n), donde |r-a_n| \leq b_1/2^{n-1}$.
		Luego, $\forall\varepsilon > 0, \exists n_0\in\N; n > n_0 \Rightarrow |a_n-r| = |r-a_n| \leq b_1/2^{n-1} < \varepsilon \Rightarrow \lim a_n = r \Rightarrow r\in\overline{G}$.

		Análogamente, $-r\in\overline{G}$, pues $G^-=\set{-a:a\in G^+}$ y $\sup G^-=0$. Como $0\in G\subset \overline{G}$, se concluye que $\R\subset\overline{G}$.

	\item Si $a\in G$, es fácil deducir inductivamente que $na\in G\ \forall n\in\Z$. Como $G\neq\set{0}$, $\exists x\in G,x\neq 0$.

		Demostremos primero que $a\in G$. Si $\inf G^+ = a \notin G$, entonces $\forall x\in R^+, \exists y \in G^+; a < y < x$ (sino, $a < x \leq \inf G^+$).
		Luego, $\exists x,y\in G^+$ tales que $a < x < a + a/2$ y $a < y < x$, por tanto, $a > a/2 > x-y \in G^+$, lo cual contradice el enunciado, y $a\in G$.

		Queda considerar si existen elementos no múltiplos de $a$.
		Sin pérdida de generalidad, consideremos $x>0$. Si $x$ no es múltiplo de $a$, entonces $\exists n\in\Z; na < x < (n+1)a \Rightarrow 0 < x-na < a$, pero $x-na\in G$ y nuevamente se llega a la contradicción.

	\item Dado $\alpha\in\R-\Q$, sea $X = \set{m+n\alpha:m,n\in\Z}$.
		Como $0 = 0+0\alpha$, $0\in X$.
		Si $m_1+n_1\alpha,m_2+n_2\alpha\in X$, entonces $(m_1-m_2)+(n_1-n_2)\alpha\in X$ porque $m_1-m_2,n_1-n_2\in\Z$.
		Luego, $X$ es un grupo aditivo de reales.

		Sea $X^+ = \set{x\in X: x>0}$, y $a = \inf X^+ >= 0$. Si $a>0$, entonces $X = \set{\pm a, \pm 2a, \ldots} = \set{na:n\in\Z}$. Sabemos que $1\in\Z$ y $\alpha\in\R-\Q$ pertenecen a $X$. Pero no existe $n\in\Z$ tal que $1=n\alpha$ o $\alpha = n\cdot 1$. Luego, es imposible que $a>0$, y se deduce que $\inf X^+ = 0$.

		Así, queda demostrado que $X$ es denso en $\R$.

\end{enumerate}\end{solucion}

	\begin{ejercicio}%!TEX root = lista3.tex

Decimos que $f:X\to\R$ es \emph{semicontinua superiormente} si
$\forall a\in X, \forall \varepsilon>0\ \exists \delta>0; x\in\openci{a}{\delta} \cap X \Rightarrow f(x) < f(a) + \varepsilon$.
Pruebe que si $K\subset\R$ es compacto con $K\neq\emptyset$ y $f:K\to\R$ es semicontinua superiormente entonces existe $b\in K$ tal que $f(x)\leq f(b)\ \forall x\in K$.\end{ejercicio}
	\begin{solucion}%!TEX root = lista2.tex

{
	\newcommand{\openci}[2]{\left(#1-#2,#1+#2\right)}
	\newcommand{\li}[1]{\mathcal{I}_{#1}}
	\newcommand{\ls}[1]{\mathcal{S}_{#1}}

	\newcommand\numberthis{\addtocounter{equation}{1}\tag{\theequation}}


	Sea $g:[0;\pinf)\to\R$ definida como $g(x)=f(x+1)-f(x)$.
	Luego, $\limite{x}{g(x)} = \limite{x}{\left[f(x+1)-f(x)\right]} = L$.
	Por definición, $\forall\varepsilon>0\ \exists A>0; x>A\Rightarrow g(x)\in\openci{L}{\varepsilon}$.

	Sea $x>A$. Existe $r\in(0;1]$ tal que $x-A-r\in\Z$. Luego,
	\begin{align*}
		\frac{f(x)}{x}
		&= \frac{\sum_{y=A+r}^{x-1}\left[f(y+1)-f(y)\right]+f(A+r)}{x}
		\\
		&= \frac{\sum_{y=A+r}^{x-1}g(y)+f(A+r)}{x}
	\end{align*}

	De $L-\varepsilon<g(y)<L+\varepsilon\ \forall y>A$ se deduce
	$$ (x-A-r)(L-\varepsilon) < \sum_{y=A+r}^{x-1}g(y) < (x-A-r)(L+\varepsilon) .$$

	Luego,
	\begin{align} \label{ineq:entorno}
		\frac{(x-A-r)(L-\varepsilon)-f(A+r)}{x} < &\frac{f(x)}{x} < \frac{(x-A-r)(L+\varepsilon)-f(A+r)}{x} \notag
		\\
		\frac{(x-A-r)}{x}\cdot(L-\varepsilon)-\frac{f(A+r)}{x} < &\frac{f(x)}{x} < \frac{(x-A-r)}{x}\cdot(L+\varepsilon)-\frac{f(A+r)}{x}
	\end{align}
	para cualquier $x>A$.

	$f$ es una función acotada en cada invervalo acotado.
	Entonces, también lo es $f(x)/x$.
	Por tanto, para todo $a\in\R$, existen $\li{a}:=\limiteinferior{x}[a]{\frac{f(x)}{x}}$ y $\ls{a}:=\limitesuperior{x}[a]{\frac{f(x)}{x}}$

	Volviendo a la desigualdad \eqref{ineq:entorno}, sea $(x_n)$ una secuencia de números en $(A,\pinf)$ tal que $\limite{n}{x_n}=\pinf$ $\limite{n}{\frac{f(x_n)}{x_n}}=\li{\pinf}$.
	Luego,
	\begin{align*} \label{ineq:li}
		\frac{f(x_n)}{x_n}
			&> \frac{(x_n-A-r)}{x_n}\cdot(L-\varepsilon)-\frac{f(A+r)}{x_n}
			\ \forall n\in\N
		\\
		\limite{n}{\frac{f(x_n)}{x_n}}
			&\geq \limite{n}{\left[\frac{(x_n-A-r)}{x_n}\cdot(L-\varepsilon)-\frac{f(A+r)}{x_n}\right]}
		\\
		\li{\pinf}
			&\geq \limite{n}{\frac{(x_n-A-r)}{x_n}}\cdot\limite{n}{(L-\varepsilon)}-\limite{n}{\frac{f(A+r)}{x_n}}
		\\
		\li{\pinf}
			&\geq L-\varepsilon .\numberthis
	\end{align*}

	Análogamente, tomando otra secuencia $(x_n)$ de números en $(A,\pinf)$ tal que $\limite{n}{x_n}=\pinf$ $\limite{n}{\frac{f(x_n)}{x_n}}=\ls{\pinf}$ obtenemos
	\begin{align*} \label{ineq:ls}
		\ls{\pinf}
			&\leq L+\varepsilon .\numberthis
	\end{align*}

	De \eqref{ineq:li} y \eqref{ineq:ls} deducimos
	$ L-\varepsilon \leq \li{\pinf} \leq \ls{\pinf} \leq L+\varepsilon \ \forall\varepsilon>0 $.
	Luego, $$ L \leq \li{\pinf} \leq \ls{\pinf} \leq L \Rightarrow \li{\pinf} = \ls{\pinf} = L \Rightarrow \limite{x}{\frac{f(x)}{x}} = L .$$
}\end{solucion}

	\begin{ejercicio}%!TEX root = lista1.tex

Sea $(a_n)$ una secuencia decreciente (no estrictamente) con $\lim a_n = 0$.
Demostrar que la serie $\sum a_n$ converge si y solamente si $\sum 2^n a_{2^n}$ converge.\end{ejercicio}
	\begin{solucion}%!TEX root = lista1.tex

Para demostrar la bicondicional, se considerarán los dos casos presentados a continuación.

~\subparagraph{Caso 1:} $\sum a_n = s$ converge.

	Como $a_n$ es decreciente, puede observarse que $2^{n-1} a_{2^n} \leq a_{2^{n-1}+1} + \ldots + a_{2^n} := b_n$. Se construye así la sucesión $(b_n)$. Es evidente que $\sum b_n = \sum a_n = s$. Como $2^na_{2^n} \leq 2b_n \forall n\in\N$, por el criterio de comparación, $\sum 2^na_{2^n}$ converge.

~\subparagraph{Caso 2:} $\sum 2^na_{2^n} = t$ converge.

	Obsérvese que $2^na_{2^n} \geq a_{2^n} + \ldots + a_{2^{n+1}-1} := c_n$.
	Al construir la sucesión $(c_n)$, se observa que $\sum c_n = \sum a_n - a_1 \Rightarrow \sum a_n = \sum c_n + a_1$.
	Como $c_n \leq 2^na_{2^n}$, por el criterio de comparación, $\sum c_n = c$ converge. Luego, $\sum a_n = c + a_1$, y también converge.\end{solucion}

\end{document}