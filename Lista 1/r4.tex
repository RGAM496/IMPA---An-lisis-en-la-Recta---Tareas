%!TEX root = lista1.tex

\hfill

\begin{enumerate}[i)]

	\item $\inf G^+ = 0$. Entonces, $x>0 \Rightarrow \exists y\in G^+; 0<y<x$ (sino, $0<x\leq \inf G^+$).

		Sean $r>0$ y $a_1\in G^+; a_1<r$. $b_1:=r-a_1>0$. Luego, $\exists c_1\in G^+; c<b_1/2$ y $\exists n\in\N; a_1+(n-1)c_1 < a+b_1/2 \leq a_1+nc_1 := a_2$. Sabiendo que $a_1,c_1\in G^+$, fácilmente se deduce que $a_2\in G^+$. Además, $a_2 = a_1+nc_1 = a_1+(n-1)c_1 + c_1 < a + b_1/2 + c_1 < a + b_1/2 + b_1/2 = a + b_1 = r$, y $b_2=r-a_2\leq b_1/2$.

		Iterando sucesivamente, obtenemos una sucesión $(a_n), donde |r-a_n| \leq b_1/2^{n-1}$.
		Luego, $\forall\varepsilon > 0, \exists n_0\in\N; n > n_0 \Rightarrow |a_n-r| = |r-a_n| \leq b_1/2^{n-1} < \varepsilon \Rightarrow \lim a_n = r \Rightarrow r\in\overline{G}$.

		Análogamente, $-r\in\overline{G}$, pues $G^-=\set{-a:a\in G^+}$ y $\sup G^-=0$. Como $0\in G\subset \overline{G}$, se concluye que $\R\subset\overline{G}$.

	\item Si $a\in G$, es fácil deducir inductivamente que $na\in G\ \forall n\in\Z$. Como $G\neq\set{0}$, $\exists x\in G,x\neq 0$.

		Demostremos primero que $a\in G$. Si $\inf G^+ = a \notin G$, entonces $\forall x\in R^+, \exists y \in G^+; a < y < x$ (sino, $a < x \leq \inf G^+$).
		Luego, $\exists x,y\in G^+$ tales que $a < x < a + a/2$ y $a < y < x$, por tanto, $a > a/2 > x-y \in G^+$, lo cual contradice el enunciado, y $a\in G$.

		Queda considerar si existen elementos no múltiplos de $a$.
		Sin pérdida de generalidad, consideremos $x>0$. Si $x$ no es múltiplo de $a$, entonces $\exists n\in\Z; na < x < (n+1)a \Rightarrow 0 < x-na < a$, pero $x-na\in G$ y nuevamente se llega a la contradicción.

	\item Dado $\alpha\in\R-\Q$, sea $X = \set{m+n\alpha:m,n\in\Z}$.
		Como $0 = 0+0\alpha$, $0\in X$.
		Si $m_1+n_1\alpha,m_2+n_2\alpha\in X$, entonces $(m_1-m_2)+(n_1-n_2)\alpha\in X$ porque $m_1-m_2,n_1-n_2\in\Z$.
		Luego, $X$ es un grupo aditivo de reales.

		Sea $X^+ = \set{x\in X: x>0}$, y $a = \inf X^+ >= 0$. Si $a>0$, entonces $X = \set{\pm a, \pm 2a, \ldots} = \set{na:n\in\Z}$. Sabemos que $1\in\Z$ y $\alpha\in\R-\Q$ pertenecen a $X$. Pero no existe $n\in\Z$ tal que $1=n\alpha$ o $\alpha = n\cdot 1$. Luego, es imposible que $a>0$, y se deduce que $\inf X^+ = 0$.

		Así, queda demostrado que $X$ es denso en $\R$.

\end{enumerate}