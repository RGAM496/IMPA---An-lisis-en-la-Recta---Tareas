%!TEX root = lista1.tex

Sea $F$ el conjunto que contiene a todas las funciones biyectivas $f:\N\to\N$. Supóngase por absurdo que $F$ es numerable, por tanto puede expresarse $F = \set{f_n: n\in\N}$.
Se procurará hallar una función biyectiva $g:\N\to\N$ tal que $g\notin F$.

Sea $A\subset\N$ un conjunto auxiliar que, en cada iteración del proceso que se describirá, acumulará valores de la imagen de $g$. Inicialmente, $A=\emptyset$. Defínase inductivamente $g$ de la siguiente forma:

\begin{enumerate}
	\item $g(1)$ es el mínimo número en $\N$ tal que $g(1) \neq f_1(1)$. Agréguese $g(1)$ a $A$. Ahora, $A = \set{g(1)}$.
	\item Para cada $n\in\N,n\geq 2$, defínase $g(n)$ como el mínimo número en $\N$ tal que $g(n)\notin A$ y $g(n)\neq f_n(n)$. Después del siguiente proceso, redefínase $A := A\cup\set{g(n)}$.
	\item Repítase el paso anterior infinitas veces.
\end{enumerate}

Es evidente que $g$ es inyectiva, por la construcción. Además, como hay infinitos valores de $n$ tales que $f_n(n) \neq n$, se puede asegurar que que cualquier valor de $n\in\N$ pertenecerá, a partir de alguna iteración, a $A$. Luego, $g$ es sobreyectiva, y por tanto, biyectiva.
Pero como $g(n) \neq f_n(n) \forall n\in\N$, $g \notin A$, lo cual lleva a la contradicción buscada.